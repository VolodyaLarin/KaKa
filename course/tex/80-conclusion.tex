\Conclusion % заключение к отчёту

В результате проведенной работы был осуществлен анализ процесса разработки компилятора для языка программирования Golang.
Были рассмотрены ключевые аспекты, включая анализ инфраструктуры, сравнительное изучение инструментов для генерации лексических и синтаксических анализаторов, а также аспекты типизации, управления ресурсами и организации пакетов в данном языке.
Данный анализ позволил сформулировать формальную задачу разработки компилятора Golang.

В ходе исследования была разработана концептуальная модель компилятора, определены методы представления пакетов и их предварительной обработки, а также спроектирована система представления типов, охватывающая грамматику языка.
Кроме того, были изучены аспекты построения абстрактного синтаксического дерева программы, спроектирована структура строителя промежуточного представления кода, и разработаны алгоритмы управления памятью, предложены сущности для хранения информации об используемых объектах и их структуры.

Выбор языка программирования C++ для реализации проекта обусловлен его мощными возможностями и поддержкой компилятором LLVM.
Разработанные программы для тестирования функциональности компилятора охватывают разнообразные языковые конструкции, а тестирование сборщика мусора с использованием инструментов анализа использования памяти позволяет удостоверить его эффективность и надежность.

Поставленные задачи были решены.
Цель работы достигнута: был разработан прототип компилятора Golang.
Полученные результаты являются основой для дальнейшей работы над развитием и оптимизацией компилятора Golang.


%В ходе работы получены следующие результаты.
%Была раскрыта сущность проблемы композиции программных интерфейсов.
%Выделены необходимые для композиции способы преобразования данных в виде операций над сущностями схемы API.
%Проведен обзор протоколов передачи данных и приведены их особенности при выполнении преобразований данных.
%
%Была выполнена классификация методов композиции программных интерфейсов в сервис-ориентированной архитектуре.
%Проведен сравнительный анализ методов и на его основании было выявлено, что полную поддержку операций над сущностями предоставляют методы композиции с помощью графа сущностей, агрегирующего хранилища и на уровне клиентского приложения.
%Методы агрегирующего хранилища и на уровне клиента не могут динамически перестраивать схему композиции.
%
%Был выполнен обзор существующих программных продуктов и на основании их анализа выявлены недостатки.
%Большинство продуктов предоставляются как облачное решение, что сужает круг применения.
%Решения Apollo Federation и Anypoint Platform предлагает полную поддержку операций над сущностями в виду поддержки композиции с помощью графа сущностей.
%
%Была проведена формализация задачи композиции программных интерфейсов в нотации IDEF0.
%
%Результаты работы показывают необходимость разработки решений для устранения следующих проблем.
%\begin{itemize}
%    \item Существует малое количество коробочных решений для композиции API с полной поддержкой операций над сущностями для протокола REST, который является одним из основных для построения веб-приложений.
%    \item Существующие решения композиции для протокола GraphQL наиболее обширны в покрытии выделенных операций над сущностями, но не обладают эффективными схемами построения множественных запросов.
%\end{itemize}
%
%Все поставленные задачи были решены.
%Таким образом цель данной работы была достигнута.
%Результаты работы могут найти применение в разработке метода композиции программных интерфейсов.

%%% Local Variables: 
%%% mode: latex
%%% TeX-master: "rpz"
%%% End: 
