\Introduction

Компилятор -- программное обеспечение, переводящее текст программы, написанный на определенном языке программирования (исходном), в машинный код, который может быть исполнен на вычислительном устройстве.
Процесс компиляции включает в себя оптимизацию кода и анализ ошибок, что способствует повышению производительности и предотвращению некоторых сбоев во время выполнения программы.
Компилятор должен быть способен обрабатывать текстовые файлы, содержащие исходный код, и создавать программу, готовую к выполнению~\cite{aho_compillers}.


Golang -- компилируемый многопоточный язык программирования, разработанный внутри компании Google.
Разработка Go началась в сентябре 2007 года, его непосредственным проектированием занимались Роберт Гризмер, Роб Пайк и Кен Томпсон.
Официально язык был представлен в ноябре 2009 года.
Go поддерживается набором компиляторов gcc, существует несколько независимых реализаций.
Язык Go разрабатывался как язык программирования для создания высокоэффективных программ, работающих на современных распределённых системах и многоядерных процессорах~\cite{bodner2024learninggo}.


\textbf{Целью работы} является разработка компилятора для языка программирования Golang.
Для достижения поставленной цели необходимо решить следующие задачи:
\begin{itemize}
    \item проанализировать грамматику языка Golang;
    \item изучить существующие подходы к анализу исходных кодов программ, а также системы для генерации низкоуровневого кода;
    \item разработать прототип компилятора, способного преобразовывать исходный код на языке Golang в машинный код.

\end{itemize}


%Проверяем как у нас работают сокращения, обозначения и определения "---
%MAX,
%\Abbrev{MAX}{Maximum ""--- максимальное значение параметра}
%API
%\Abbrev{API}{application programming interface ""--- внешний интерфейс взаимодействия с приложением}
%с обратным прокси.
%\Define{Обратный прокси}{тип прокси-сервера, который ретранслирует}



