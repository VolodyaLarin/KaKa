\chapter{Технологический раздел}
\label{cha:impl}

\section{Выбор средств реализации}

В ходе разработки было решено использовать язык программирования C++, так как он предоставляет широкие возможности для реализации приложений.
Язык программирования C++ широко поддерживается компилятором LLVM, что обеспечивает высокую производительность и эффективность разработки.
Кроме того, ANTLR4, используемый для генерации кода лексера и парсера, также порождает исходный код на C++, что позволяет легко интегрировать сгенерированный код.
Также код на C++ легко компонуется с компилируемыми бекендом LLVM программами, благодаря поддержке соглашения вызовов cdecl, что облегчает процесс разработки и интеграции.
Для обеспечения удобного импорта и экспорта данных о структуре пакетов была выбрана библиотека Boost, обладающая нужными инструментами для работы с данными.
Для написания кода были использованы интегрированные среды разработки JetBrains CLion и GoLand.

\section*{Структура проекта}


На рисунке~\ref{fig:code-usage} представлен граф включаемых заголовочных файлов для main.cpp.

\begin{figure}[h]
    \centering

    \includesvg[width=\textwidth]{plantuml/export/code-usage}

    \caption{Граф включаемых заголовочных файлов для main.cpp}
    \label{fig:code-usage}
\end{figure}


\section{Тестирования компилятора}

\subsection*{Тестирование генерации кода}

Были созданы программы, которые соответствуют измененной грамматике языка.
Они предназначены для проверки правильности работы различных конструкций языка.
Данные программы охватывают следующие сценарии:
\begin{itemize}
	\item создание переменных и констант;
	\item ввод и вывод;
	\item работа с целыми и вещественными числами;
	\item работа с константными строками;
	\item создание функций и работа с рекурсией;
	\item создание ветвлений программы;
	\item создание циклов, выход из цикла и пропуск итерации цикла;
	\item работа с массивами фиксированного размера;
	\item работа со структурами, создание методов структур;
	\item создание интерфейсов;
	\item работа с указателями;
	\item логические операции, ленивые логические вычисления;
	\item преобразование типов.
\end{itemize}


В листинге~\ref{lst:prog_example_interfaces} представлен пример работы с интерфейсами.

\begin{lstlisting}[label=lst:prog_example_interfaces,language=go,basicstyle=\scriptsize,numberstyle=\tiny,caption={Пример программы использования интерфейсов}]
package main
import "fmt"
type geometry interface {
	area() float64
	perim() float64
}
type rect struct {
	width  float64
	height float64
}
type circle struct {
	radius float64
}
func (r rect) area() float64 {
	return r.width * r.height
}
func (r rect) perim() float64 {
	return 2*r.width + 2*r.height
}
func (c circle) area() float64 {
	return 3.14 * c.radius * c.radius
}
func (c circle) perim() float64 {
	return 2 * 3.14 * c.radius
}
func measure(g geometry) {
	fmt.Printf("Area: %g cm^2, perim: %g cm\n", g.area(), g.perim())
}
func main() {
	var r rect
	r.width = float64(3)
	r.height = float64(4)
	var c circle
	c.radius = float64(5)
	var g1 geometry = r
	measure(g1)
	var g2 geometry = c
	measure(g2)
}
\end{lstlisting}

Было обнаружено различие в работе данного и оригинального компилятора, заключающеюся в форматировании булевых значений.
Оно связано с тем, что используются разные реализации функции printf.
Данное различие несущественно и может быть опущено из рассмотрения.
Компилятор справился со всеми описанными задачами успешно.

\subsection*{Тестирование работы с памятью}

Для тестирования сборщика мусора был использован инструмент Valgrind с использованием предыдущих программ.

Рассмотрим следующий пример, представленный на листинге~\ref{lst:prog_example_gc}.
В программе создается связанный список из 1000 узлов.
Каждый узел занимает примерно 390 КиБ.
Затем голова списка обнуляется, чтобы ссылки на все узлы стали недоступными для дальнейшего использования.
Затем создается связанный список из 200 узлов.
Данные операции были повторены 10 раз.
Результат работы сборщика мусора можно оценить, наблюдая изменения в использовании памяти программой.

\begin{lstlisting}[label=lst:prog_example_gc,language=go,basicstyle=\scriptsize,numberstyle=\tiny,caption={Программа для тестирования сборщика мусора}]
package main
type Node struct {
	data     [100000]int
	nextNode *Node
}
func newNode(next *Node) *Node {
	var node Node
	node.nextNode = next
	return &node
}
func main() {
	var head *Node = nil
	for i := 0; i < 10; i++ {
		for i := 0; i < 1000; i++ {
			head = newNode(head)
		}
		head = nil
		for i := 0; i < 200; i++ {
			head = newNode(head)
		}
		head = nil
	}
}
\end{lstlisting}

Инструмент Valgrind не выявил ошибок в использовании памяти.
Также был получен график использования памяти, представленный на рисунке~\ref{fig:valgrind-massif}.
Из данной зависимости видно, что сборщик мусора высвобождал неиспользуемые узлы.
Следовательно, сборщик мусора работает корректно.

\begin{figure}[h]
	\centering

	\includesvg[width=\textwidth]{plantuml/export/valgrind_massif}

	\caption{Зависимость используемой памяти от времени для программы, представленной на листинге~\ref{lst:prog_example_gc}}
	\label{fig:valgrind-massif}
\end{figure}


%В данном разделе описано изготовление и требование всячины. Кстати,
%в Latex нужно эскейпить подчёркивание (писать <<\verb|some\_function|>> для \Code{some\_function}).
%
%\ifPDFTeX
%Для вставки кода есть пакет \Code{listings}. К сожалению, пакет \Code{listings} всё ещё
%работает криво при появлении в листинге русских букв и кодировке исходников utf-8.
%В данном примере он (увы) на лету конвертируется в koi-8 в ходе сборки pdf.
%
%Есть альтернатива \Code{listingsutf8}, однако она работает лишь с
%\Code{\textbackslash{}lstinputlisting}, но не с окружением \Code{\textbackslash{}lstlisting}
%
%Вот так можно вставлять псевдокод (питоноподобный язык определен в \Code{listings.inc.tex}):
%
%\begin{lstlisting}[style=pseudocode,caption={Алгоритм оценки дипломных работ}]
%def EvaluateDiplomas():
%    for each student in Masters:
%        student.Mark := 5
%    for each student in Engineers:
%        if Good(student):
%            student.Mark := 5
%        else:
%            student.Mark := 4
%\end{lstlisting}
%
%Еще в шаблоне определен псевдоязык для BNF:
%
%\begin{lstlisting}[style=grammar,basicstyle=\small,caption={Грамматика}]
%  ifstmt -> "if" "(" expression ")" stmt |
%            "if" "(" expression ")" stmt1 "else" stmt2
%  number -> digit digit*
%\end{lstlisting}
%
%В листинге~\ref{lst:sample01} работают русские буквы. Сильная магия. Однако, работает
%только во включаемых файлах, прямо в \TeX{} нельзя.
%
%% Обратите внимание, что включается не ../src/..., а inc/src/...
%% В Makefile есть соответствующее правило для inc/src/*,
%% которое копирует исходные файлы из ../src и конвертирует из UTF-8 в KOI8-R.
%% Кстати, поэтому использовать можно только русские буквы и ASCII,
%% весь остальной UTF-8 вроде CJK и египетских иероглифов -- нельзя.
%
%\lstinputlisting[language=C,caption=Пример (\Code{test.c}),label=lst:sample01]{inc/src/test.c}
%
%\else
%
%Для вставки кода есть пакет \texttt{minted}. Он хорош всем кроме: необходимости Python (есть во всех нормальных (нет, Windows, я не про тебя) ОС) и Pygments и того, что нормально работает лишь в \XeLaTeX.
%
%\ifdefined\NoMinted
%Но к сожалению, у вас, по-видимому, не установлен Python или pygmentize.
%\else
%Можно пользоваться расширенным BFN:
%
%\begin{listing}[H]
%\begin{ebnfcode}
% letter = "A" | "B" | "C" | "D" | "E" | "F" | "G"
%       | "H" | "I" | "J" | "K" | "L" | "M" | "N"
%       | "O" | "P" | "Q" | "R" | "S" | "T" | "U"
%       | "V" | "W" | "X" | "Y" | "Z" ;
%digit = "0" | "1" | "2" | "3" | "4" | "5" | "6" | "7" | "8" | "9" ;
%symbol = "[" | "]" | "{" | "}" | "(" | ")" | "<" | ">"
%       | "'" | '"' | "=" | "|" | "." | "," | ";" ;
%character = letter | digit | symbol | "_" ;
%
%identifier = letter , { letter | digit | "_" } ;
%terminal = "'" , character , { character } , "'"
%         | '"' , character , { character } , '"' ;
%
%lhs = identifier ;
%rhs = identifier
%     | terminal
%     | "[" , rhs , "]"
%     | "{" , rhs , "}"
%     | "(" , rhs , ")"
%     | rhs , "|" , rhs
%     | rhs , "," , rhs ;
%
%rule = lhs , "=" , rhs , ";" ;
%grammar = { rule } ;
%\end{ebnfcode}
%\caption{EBNF определённый через EBNF}
%\label{lst:ebnf}
%\end{listing}
%
%А вот в листинге \ref{lst:c} на языке C работают русские комменты. Спасибо Pygments и Minted за это.
%
%\begin{listing}[H]
%\cfile{inc/src/test.c}
%\caption{Пример — test.c}
%\end{listing}
%\label{lst:c}
%
%\fi
%\fi
%% Для вставки реального кода лучше использовать \texttt{\textbackslash lstinputlisting} (который понимает
%% UTF8) и стили \Code{realcode} либо \Code{simplecode} (в зависимости от размера куска).
%
%
%
%
%Можно также использовать окружение \Code{verbatim}, если \Code{listings} чем-то не
%устраивает. Только следует помнить, что табы в нём <<съедаются>>. Существует так же команда \Code{\textbackslash{}verbatiminput} для вставки файла.
%
%\begin{verbatim}
%a_b = a + b; // русский комментарий
%if (a_b > 0)
%    a_b = 0;
%\end{verbatim}
%
%%%% Local Variables:
%%%% mode: latex
%%%% TeX-master: "rpz"
%%%% End:
