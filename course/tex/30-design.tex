\chapter{Конструкторский раздел}
\label{cha:design}


\section{Концептуальная модель}


На рисунке~\ref{fig:idef-a0-full} представлен схема процесса компиляции языка программирования Golang.

\begin{figure}[h]
    \centering

    \includesvg[width=\textwidth]{plantuml/export/idef/02_A0}

    \caption{Детализированная IDEF0 диаграмма компилятора языка программирования Golang}
    \label{fig:idef-a0-full}
\end{figure}

Данная диаграмма разделяет компилятор на ряд компонентов.
Анализатор пакетов принимает путь до пакета и использует стандартную библиотеку, передавая далее исходный код.
Лексер трансформирует исходный код в массив лексем, а парсер на основе данного массива строит AST-дерево.
Строитель промежуточного представления IR использует AST-дерево и информацию о пакетах для создания промежуточного представления.
Затем строитель исполняемого кода конвертирует промежуточное представление в объектный файл с помощью LLVM.
Линковщик собирает объектные файлы и формирует исполняемый файл.


\section{Система пакетов}

Анализатор пакетов строит дерево зависимостей и поочередно от листов к корню собирает пакеты.
Корневым пакетом является исполняемый пакет, имеет имя \Code{main} и функцию \Code{main}.

\subsection*{Представление пакета}

Было разработано представление пакета и базовых типов, представленное на рисунке~\ref{fig:package-pres}.
Пакет содержит имя пакета, его хэш-имя, которое будет использоваться для декларации функций и констант в объектных файлах для их линковки.
Представление также содержит хэш-таблицы на объявленные типы, функции, константы.


\begin{figure}[h]
    \centering

    \includesvg[width=0.7\textwidth]{plantuml/export/package-pres}

    \caption{Представление пакета}
    \label{fig:package-pres}
\end{figure}


Была предложена поддержка следующих базовых типов.
IntegerType -- целое число длины size бит, обладающее знаком согласно флагу isSigned.
FloatType -- число с плавающей запятой длины size бит.
StructType -- структура, содержащая поля с именами FieldNames и типами FieldsTypes.
ArrayType -- массив, содержащий элементы типа type.
PointerType -- указатель на тип type.
FunctionType -- функция, имеющая аргументы типы args и возвращаемый тип retType.

Данного представления достаточно, чтобы поддерживать базовые типы языка: bool, byte, float32, float64, int, int16, int32, int64, int8, rune, string, uint, uint16, uint32, uint64, uint8, -- создавать пользовательские структуры, интерфейсы, функции, методы и экспортировать их из пакета
Пример пакета стандартной библиотеки builtin представлен в приложении~\ref{cha:appendix_builtin}.

\subsection*{Дерево зависимостей пакетов}


Алгоритм~\ref{alg:A_DependencyTree} реализует построение дерева зависимостей для пакетов.
Он начинает с корневого пакета и использует обход графа зависимостей для добавления зависимостей в дерево.
Каждая зависимость обрабатывается только один раз, что позволяет избежать циклических зависимостей и гарантирует эффективное построение дерева.

\begin{breakablealgorithm}
    \caption{Формирования дерева зависимостей}
    \label{alg:A_DependencyTree}

    \begin{algorithmic}[1]
        \Function{$BuildDependencyTree$}{ $rootPackage$}
            \Statex $\triangleright$ $rootPackage$ "--- корней пакет
            \Statex
            \State $dependencyTree \leftarrow$ создать граф
            \State $queue \leftarrow$ создать очередь
            \State добавить $rootPackage$ в очередь $queue$
            \While{очередь $queue$ не пуста}
                \State $currentPackage \leftarrow$ взять из очереди $queue$
                \State $dependencies \leftarrow$ получить массив зависимостей пакета $currentPackage$
                \ForAll{$dependency \in dependencies$}
                    \State добавить ребро $currentPackage \rightarrow dependency$ в граф $dependencyTree$
                    \If{$dependency \notin dependencyTree$}
                        \State добавить $dependency$ в очередь $queue$
                    \EndIf
                \EndFor
            \EndWhile
            \State \Return $dependencyTree$
        \EndFunction
    \end{algorithmic}
\end{breakablealgorithm}



\section{Построение анализаторов исходного кода}


В данной работе лексический и синтаксический анализаторы создаются при помощи ANTLR.
На вход принимается грамматика языка программирования в формате ANTLR4.
Грамматика приведена в приложении~\ref{cha:appendix1}.
Она состоит из объявления ключевых слов и правил вывода.


В результате работы ANTLR4 генерируются файлы, содержащие классы лексического и синтаксического анализаторов, а также необходимые вспомогательные файлы и классы для их функционирования.
Кроме того, создаются шаблоны классов для обхода дерева разбора, полученного в результате работы парсера.

Лексер принимает на вход текст программы, преобразованный в поток символов, и выдает поток токенов, который передается на вход парсера.
После выполнения парсера создается дерево разбора.
Любые ошибки, возникающие в процессе работы лексера и парсера, выводятся в стандартный поток ввода-вывода.


Абстрактное синтаксическое дерево (AST) предоставляет структурированное представление программы.
Его можно обойти двумя основными способами: с помощью паттерна Listener и паттерна Visitor.

Паттерн Listener позволяет обходить дерево в глубину, вызывая обработчики соответствующих событий при входе и выходе из узла дерева.
Когда происходит обход дерева, при входе в узел вызывается соответствующий обработчик, а при выходе - другой обработчик.
Это позволяет реализовать обработку узлов на основе их типов и состояний.

Паттерн Visitor предоставляет более гибкий способ обхода дерева.
Он позволяет определить, какие узлы и в каком порядке нужно посетить.
Для каждого типа узла в дереве реализуется метод его посещения.
Обход начинается с точки входа в программу, обычно с корневого узла.
Каждый узел посещается в заданном порядке, что позволяет реализовать различные алгоритмы обработки AST-дерева.


\section{Построение промежуточного представления (IR)}


В данной работе был использован паттерн Visitor для обхода абстрактного синтаксического дерева.
Данный подход обеспечивает гибкость и позволяет точно контролировать порядок посещения узлов дерева, что удобно при реализации различных анализов и преобразований.

На рисунке~\ref{fig:builder-ir} представлена диаграмма классов, определяющая структуру строителей и посетителей для компилятора языка программирования Golang.


\begin{figure}[h]
    \centering

    \includesvg[width=1\textwidth]{plantuml/export/ir-builder}

    \caption{Структура строителя IR представления}
    \label{fig:builder-ir}
\end{figure}


В ней есть два основных строителя: LlvmBuilder, который отвечает за создание LLVM IR кода, и GoIrBuilder, который является адаптером для LlvmBuilder и предоставляет интерфейс для создания промежуточного представления кода Golang.
GoIrBuilder обладает достаточным интерфейсом для разработки компилятора данного языка программирования.
Класс GoIrVisitor представляет собой посетителя, который обходит абстрактное синтаксическое дерево (класс AST) и выполняет различные операции над его узлами.
Он расширяет базовый посетитель BaseVisitor, чтобы обеспечить специфическую функциональность для языка Golang.
Связь между GoIrVisitor и GoIrBuilder показывает, что посетитель использует строитель для создания кода внутри методов обхода AST-дерева.


\section{Алгоритмы управления памятью}

\subsection*{Формирование теневого стека}


Для отслеживания достижимых ссылок из кода программы предлагается использовать теневой стек.
При вызове подпрограмм предлагается добавлять новый слой в него, а при возврате управления -- удалять данный слой.
В слое хранятся ссылки на выделенные объекты.

Для каждого выделенного объекта предлагается хранить его разметку, где указывается местоположения указателей объекта.
Таким образом можно получить ссылки объекта, зная только информацию о его разметке в памяти.
Также для дальнейшей очистки неиспользуемых объектов предлагается использовать флаг об использовании.

На рисунке~\ref{fig:heap-pres} показано представление в памяти теневых стека и кучи.

\begin{figure}[h]
    \centering

    \includesvg[width=\textwidth]{plantuml/export/heap-pres}

    \caption{Представление в памяти теневых стека и кучи}
    \label{fig:heap-pres}
\end{figure}


Алгоритм~\ref{alg:A_Allocate} предназначен для выделения памяти под новый объект в куче.
На вход функция алгоритма принимает размер выделяемого объекта в байтах и массив ссылок на другие объекты, указывающих на начало указателей в объекте.
Внутри функции сначала выделяется блок памяти заданного размера и очищается.
Затем адрес выделенной памяти добавляется в конец глобального теневого стека, а в хэш-таблицу кучи по адресу объекта добавляется информация о его разметке.
В конце функция возвращает указатель на выделенную память.

\begin{breakablealgorithm}
    \caption{Создание нового объекта}
    \label{alg:A_Allocate}

    \begin{algorithmic}[1]
        \Function{Allocate}{size, links}
            \Statex $\triangleright$ $size$ "--- размер выделяемого типа в байтах
            \Statex $\triangleright$ $links$ "--- массив позиций в типе, указывающих на начало ссылок на иные объекты
            \Statex $\triangleright$ $shadowStack$ "--- глобальный теневой стек
            \Statex $\triangleright$ $shadowHeap$ "--- глобальная хэш-таблица: адрес памяти / информация о ссылках
            \Statex
            \State $ptr \leftarrow$ выделить $size$ байт и очистить их
            \State добавить адрес $ptr$ в последний блок теневого стека $shadowStack$
            \State добавить по ключу $ptr$ значение $links$ в таблицу $shadowHeap$
            \State \Return $ptr$
        \EndFunction
    \end{algorithmic}
\end{breakablealgorithm}

\subsection*{Сбор неиспользуемой памяти}


Алгоритм~\ref{alg:A_GC} предназначен для освобождения памяти, которая больше не используется программой.
На вход алгоритму подается массив указателей ptrs, содержащий адреса дополнительно сохраняемых объектов.
В данном случае -- это значения, возвращаемые последней функцией, так как они могли еще не сохраниться в какой-либо переменной на стеке и могут быть очищены.
Происходит проход по каждому сохраняемого объекту и вызывается функция Mark, которая помечает объекты, как используемые.
Затем выполняется проход по глобальному теневому стеку shadowStack, чтобы пометить объекты стека как используемые.
После завершения данного процесса все объекты, на которые не указывают ни массив ptrs, ни глобальный теневой стек, освобождаются.
Освобождение памяти происходит путем удаления объекта из глобальной хэш-таблицы shadowHeap и отправка запроса на удаление алокатору.

\begin{breakablealgorithm}
    \caption{Сбор неиспользуемой памяти}
    \label{alg:A_GC}
    \begin{algorithmic}[1]
        \Function{$GarbageCollection$}{$ptr$}
            \Statex $\triangleright$ $ptrs$ "--- массив указателей на сохраняемые объекты
            \Statex $\triangleright$ $shadowHeap$ "--- глобальная хэш-таблица: адрес памяти / информация о ссылках
            \Statex $\triangleright$ $shadowStack$ "--- глобальный теневой стек
            \Statex
            \ForAll{$ptr \in ptrs$}
                \State $Mark(ptr)$
            \EndFor
            \ForAll{$layer \in shadowStack$}
                \ForAll{$ptr \in layer$}
                    \State $Mark(ptr)$
                \EndFor
            \EndFor

            \ForAll{$ptr \in shadowHeap$, которые не помечены как используемые}
                \State Освободить память $ptr$
            \EndFor
        \EndFunction
    \end{algorithmic}
\end{breakablealgorithm}

Алгоритм~\ref{alg:A_Mark} предназначен для пометки объекта и всех объектов, на которые он ссылается, как используемые.
На вход алгоритму передается указатель на объект, который необходимо пометить.
Сначала проверяется, не является ли указатель пустым.
Затем из глобальной хэш-таблицы кучи получается информация о ссылках объекта и для каждой ссылки вызывается рекурсивно данная процедура.

\begin{breakablealgorithm}
    \caption{Пометить объект и всех объектов, на которые он ссылается}
    \label{alg:A_Mark}

    \begin{algorithmic}[1]
        \Function{$Mark$}{$ptr$}
            \Statex $\triangleright$ $ptr$ "--- указатель на раскрашиваемый объект
            \Statex $\triangleright$ $shadowHeap$ "--- глобальная хэш-таблица: адрес памяти / информация о ссылках
            \Statex
            \If{$ptr$ -- пустой}
                \State \Return
            \EndIf
            \State $links \leftarrow$ получить значение из $shadowHeap$ по ключу $ptr$
            \If{не удалось получить значение $links$}
                \State вывести сообщение о наличии <<дикого>> указателя
                \State \Return
            \EndIf
            \If{$ptr$ помечен как используемый}
                \State \Return
            \EndIf
            \State пометить $ptr$ как используемый
            \ForAll{$link \in links$}
                \State $linkPtr \leftarrow$ получить значение по адресу $ptr + link$
                \State $Mark(linkPtr)$
            \EndFor

            \State \Return

        \EndFunction
    \end{algorithmic}
\end{breakablealgorithm}

С помощью предложенных алгоритмов реализуются основные процессы выделения и утилизации памяти.

\section*{Выводы}

Была разработана концептуальная модель компилятора языка программирования Golang.
Предложены способы представления пакетов и их предварительной обработки.
Спроектирована система представления типов, покрывающая грамматику языка.
Рассмотрены вопросы построения абстрактного синтаксического дерева программы.
Структура строителя промежуточного представления кода была спроектирована.
Были описаны алгоритмы управления памяти и предложены сущности для хранения информации об используемых объектах и их структуры.
