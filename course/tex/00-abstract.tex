% Также можно использовать \Referat, как в оригинале
%\begin{abstract}
\chapter{РЕФЕРАТ}%

Отчет содержит \pageref{LastPage}\,стр.%
\ifnum \totfig >0
, \totfig~рис.%
\fi
\ifnum \tottab >0
, \tottab~табл.%
\fi
%
\ifnum \totbib >0
, \totbib~источн.%
\fi
%
\ifnum \totapp >0
, \totapp~прил.%
\else
.%
\fi


Ключевые слова: компиляция, сбор мусора, Go, Golang, LLVM, AST, ANTLR, исполняемый код, анализатор, строитель дерева.

Предмет исследования~-- компилятор.
Объект исследования~--построение компилятора языка GoLang.


Компилятор – это программное средство, которое преобразует исходный текст программы на определенном языке программирования в машинный код для исполнения на целевой архитектуре.
Golang – это многопоточный язык программирования, разработанный Google для создания распределенных систем.
Был проведен анализ инфраструктуры разработки компилятора Golang, включающий сравнительное изучение инструментов для генерации лексических и синтаксических анализаторов, а также аспектов типизации и управления ресурсами.
Исследование привело к разработке концептуальной модели компилятора, методов обработки пакетов, системы представления типов и структуры построения промежуточного представления кода.
Для реализации проекта были выбраны LLVM, ANTLR4, язык программирования C++.
Разработанные программы для тестирования компилятора охватывают разнообразные языковые конструкции.

Полученные результаты являются основой для будущей работы над развитием и оптимизацией компилятора Golang.
Было проведено тестирование функциональности компилятора и сборщика мусора.



%\end{abstract}

%%% Local Variables: 
%%% mode: latex
%%% TeX-master: "rpz"
%%% End: 
